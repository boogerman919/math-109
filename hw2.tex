\documentclass{article}
\usepackage{amsfonts, amsmath, amssymb, amsthm} % Math notations imported
\usepackage{enumitem}

% If you want to use another style of headers, uncomment (hotkey: ctrl + /) these 7 lines, and comment out "\maketitle" below
% \usepackage{fancyhdr} % Import package
% \pagestyle{fancy} % using the "fancy" pagestyle
% \fancyhf{} % clear out original headers and footers
% \lhead{Math 109 HW 1} % left header, rest is self-explanatory
% \rhead{(your name)}
% \lfoot{(date when hw is due)}
% \rfoot{Page \thepage}

% Some basic theorem environments set up
\newtheorem{thm}{Theorem}
\newtheorem{prop}[thm]{Proposition}
\newtheorem{cor}[thm]{Corollary}

% title information
\title{Math 109 HW 2}
\author{Ray Tsai}
\date{10/10/2022}

% main content
\begin{document} 

% placing title information; comment out if using fancyhdr
\maketitle 

\begin{enumerate}
% Q1
\item 
\begin{enumerate}
    \item
    Let $x$ be a positive integer, we will show that $x + 1$ is a positive integer.
    \item
    Let $x, y$ be some odd integers, we will show that $x + y$ is an even integer.
    
\end{enumerate}

% Q2
\item \begin{prop}
    For all integers $a, b, c$, we have that $a(b+c) = ab+ac$.
    \end{prop}
    \begin{proof} \
    \begin{center}
    \begin{tabular}{|c|c|c|}
    \hline
    Step & Know & Reason \\ \hline
    P & Let $a, b, c$ be integers & Assumption \\ \hline
    Q & $a(b + c) = ab + ac$ & Distribution Law \\ \hline
\end{tabular}
\end{center}
\end{proof}

% Q3
\item \begin{prop}
    If $m$ is an even integer and $n$ is an odd integer, then $m + n$ is an odd integer.
    \end{prop}
    \begin{proof} \
    \begin{center}
    \begin{tabular}{|c|c|c|}
    \hline
    Step & Know & Reason \\ \hline
    P & Let $m$ be $2k$ and $n$ be $2l + 1$, where $k, l$ are some integers & Assumption \\ \hline
    P1 & $m + n = 2k + 2l + 1$ & Substitution of Equals \\ \hline
    P2 & $= 2(k + l) + 1$ & Distribution Law \\ \hline
    P3 & Let $k + l$ be some integer $h$ & Assumption \\ \hline
    P4 & $m + n = 2h + 1$ & Substitution of Equals \\ \hline
    Q & Thus, $m + n$ is odd & Definition of $odd$ \\ \hline 
\end{tabular}
\end{center}
\end{proof} 

% Q4
\item 
\begin{prop}
For all integers $a, b, c, d$, we have that $(a + b)(c + d) = ac + ad + bc + bd$
\end{prop}
\begin{proof}
Let $a, b, c ,d$ be integers. We will show that $(a + b)(c + d) = ac + ad + bc + bd$. \\\\
By distribution law 
\begin{align}
(a + b)(c + d) &= a(c + d) + b(c + d) \\ &= ac + ad + bc + bd
\end{align}
\end{proof}

\begin{cor}
If $a, b$ are integers, then $(a + b)^2 = a^2 + 2ab + b^2$.
\end{cor}
\begin{proof}
Let $a, b$ be integers. We will show that $(a + b)^2 = a^2 + 2ab + b^2$. \\\\
By HW2 Q4, we know that $(a + b)(c + d) = ac + ad + bc + bd$.
\begin{align}
(a + b) ^2 &= (a + b)(a + b) \\ &= aa + ab + ab + bb \\ &= a^2 + 2ab + b^2
\end{align}
\end{proof}

% Q5
\item 
\begin{prop}
If $a$ is an odd integer, then $a^2$ is also an odd integer.
\end{prop}

\begin{proof}
Let $a = 2k +1$ for some integer $k$. We will show that $a^2$ is an odd integer. 

By HW2 Q4 Corollary, we know that $(a + b)^2 = a^2 + 2ab + b^2$
\begin{align}
a^2 &= (2k + 1)^2 \\ &= 4k^2 + 4k + 1
\end{align}
By Distribution Law
\begin{align}
4k^2 + 4k + 1 &= 2(2k^2 + 2k) +1
\end{align}

Let $2k^2 + 2k$ be an integer $m$
\begin{align}
a^2 &= 2(2k^2 + 2k) +1 \\ &= 2m + 1
\end{align}
Thus, $a^2$ is an odd number by definition.
\end{proof}

% Q6
\item \begin{prop}
If $a, b$ are integers with $a + b = a$, then $b = 0$.
\end{prop}
\begin{proof}
Let $a,b$ be some integers. We will show that if $a + b = a$, then $b = 0$.
\begin{gather}
a + b = a \\ 
a + b + (-a) = a + (-a) \\
\end{gather}
By Distribution Law
\begin{gather}
(1-1)a + b = (1-1)a \\
0\cdot a + b = 0\cdot a \\
b = 0
\end{gather}
Thus, if $a + b = a$, then $b = 0$.
\end{proof}

% Q7
\item
\begin{prop}
For all positive integers $a, b$, we have that $ab$ is positive.
\end{prop}
\begin{proof}
Let $a, b$ be integers where $a, b> 0$. We will show that $ab$ is positive.
\begin{gather}
a > 0\quad b > 0
\end{gather}
By HW2 Fact 11
\begin{gather}
ab > 0\cdot b \\
ab > 0
\end{gather}
Thus, by definition, $ab$ is positive by if $a,b$ are both positive integers.
\end{proof}

% Q8
\item
\begin{prop}
If $a$ is a positive integer and $b$ is a negative integer, then $ab$ is negative.
\end{prop}
\begin{proof}
Let $a, b$ be integers where $a > 0, b < 0$. We will show that $ab$ is negative.\\
\begin{gather}
a > 0\quad b < 0
\end{gather}
By HW2 Fact 12
\begin{gather}
-b > 0
\end{gather}
By HW2 Fact 11
\begin{gather}
a(-b) > 0\cdot -b = 0
\end{gather}
By Associative Law of Multiplication
\begin{gather}
-ab > 0
\end{gather}
Again, by HW2 Fact 12
\begin{gather}
-(-ab) = ab < 0
\end{gather}
Thus, by definition, $ab$ is negative if $a$ is a positive integer and $b$ is a negative integer.
\end{proof}

% Q9
\item
\begin{prop}
If $a$ is a negative integer and $b, c$ are integers with $b > c$, then $ab < ac$.
\end{prop}
\begin{proof}
Let $a, b, c$ be integers where $a < 0, b > c$. We will show that $ab < ac$.
\begin{gather}
a < 0\quad b > c
\end{gather}
By HW2 Fact 12
\begin{gather}
-a > 0
\end{gather}
By HW2 Fact 11
\begin{gather}
(-a)b > (-a)c
\end{gather}
By Associative Law of Multiplication
\begin{gather}
-ab > -ac
\end{gather}
Again, by HW2 Fact 12
\begin{gather}
ab < ac
\end{gather}
Thus, $ab < ac$ if $a$ is a negative integer and $b > c$.
\end{proof}
    
\end{enumerate}
\end{document}